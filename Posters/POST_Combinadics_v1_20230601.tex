\documentclass[a0paper,portrait,final]{baposter}

\usepackage{amsmath}
\usepackage{amssymb}
\usepackage{amsthm}
\usepackage{graphicx}
\usepackage{hyperref}
\usepackage{subcaption}
\usepackage{float}
\usepackage{tikz}
\usetikzlibrary{decorations.pathreplacing,positioning,calc,intersections,3d,shapes.geometric,shapes,chains,math,fit,backgrounds}
\usepackage{colortbl}
\usepackage{algorithm}
\usepackage[noend]{algpseudocode}
\usepackage{xcolor}

\usepackage{pdfpages}
\usepackage{multicol}

\newcommand{\abs}[1]{\left \vert #1 \right \vert}
\newcommand{\norm}[1]{\left \Vert #1 \right \Vert}
\newcommand{\order}[1]{\mathcal{O} \left ( #1 \right )}
\newcommand{\set}[1]{\left \{ #1 \right \}}
\newcommand{\Set}[2]{\left \{ #1 \ \middle \vert \ #2 \right \}}
\newcommand{\vmat}[1]{\begin{vmatrix} #1 \end{vmatrix}}
\DeclareMathOperator{\sign}{sign}
\DeclareMathOperator{\conv}{conv}
\DeclareMathOperator{\Span}{span}
\DeclareMathOperator{\diag}{diag}

\newcommand{\bbn}{\mathbb{N}}
\newcommand{\bbz}{\mathbb{Z}}
\newcommand{\bbq}{\mathbb{Q}}
\newcommand{\bbr}{\mathbb{R}}
\newcommand{\bbc}{\mathbb{C}}
\newcommand{\bbf}{\mathbb{F}}

\newtheorem{conj}{Conjecture}
\newtheorem{definition}{Definition}
\newtheorem{proposition}{Proposition}
\newtheorem{lemma}{Lemma}
\newtheorem{corollary}{Corollary}
\newtheorem{theorem}{Theorem}

\definecolor{darkgreen}{rgb}{0,0.6,0.1}

\let\vec\mathbf

\begin{document}

\begin{poster}{
grid=true,
eyecatcher=true,
%=== Column spacing ===
% columns=2,
colspacing=1em,
%=== Color style ===
bgColorOne=white,
bgColorTwo=white,
borderColor=blue,
headerColorOne=blue,
headerColorTwo=blue,
headerFontColor=white,
boxColorOne=white,
%boxColorTwo=lightblue,
%=== Format of textbox ===
textborder=rounded,
%=== Format of text header ===
headerborder=closed,
headerheight=0.1\textheight,
%  textfont=\sc, An example of changing the text font
headershape=roundedright,
headershade=shadelr,
headerfont=\Large\bf\textsc, %Sans Serif
textfont={\setlength{\parindent}{1.5em}},
boxshade=plain,
%  background=shade-tb,
background=plain,
linewidth=2pt%
}

{\centering\bf\textsc{Combinadics adjacency matrix}\vspace{0.2em}}
{\textsc{Conor McCoid\\ \small{Universit\'e Laval}}}
% nb: add logo to right

\headerbox{Combinadics}{name=def,column=0,row=0,span=1,height=0.19}{
The combinatorial number system, or combinadics, is a way to enumerate all $m$-combinations.
Each combination of $m$ natural numbers is given a ranking from 0 to $\binom{n}{m} - 1$.

The ranking $N(J)$ of an $m$--combination $J=\set{c_1,\dots,c_m}$, arranged in lexicographical order, of natural numbers $\set{0,\dots,n-1}$ is equal to
\begin{equation*}
N(J) = \binom{c_m}{m} + \dots + \binom{c_1}{1}.
\end{equation*}
}


\headerbox{Unranking}{name=rank,column=1,row=0,span=1,height=0.19}{
Generating $J$ from $N(J)$, called unranking, can be achieved through a greedy algorithm using the same coefficients.
Let $c$ be the largest number in $\set{0,\dots,n-1}$ such that $N(J) \geq \binom{c}{m}$.
Then $c \in J$.
Subtract $\binom{c}{m}$ from $N(J)$ and repeat, now considering the numbers $\set{0,\dots,c-1}$ and the $(m-1)$--combination $J \setminus \set{c}$.
}

\headerbox{Important lemma}{name=lemma,column=2,row=0,span=1,height=0.08}{
\begin{lemma} \label{lem: rank sum}
For $m$-combination $J$
\begin{equation*}
N(J^\complement) + N(J) = \binom{n}{m} - 1
\end{equation*}
\end{lemma}
}

\headerbox{Anti-transpose}{name=unrank,column=2,row=0,span=1,height=0.1,below=lemma}{
The anti-transpose of a matrix $A$, denoted $A^\tau$, is its reflection over its northeast-to-southwest diagonal.
That is, if $A \in \bbr^{m \times m}$ then $\left ( A^\tau \right )_{i,j} = \left ( A \right )_{m-j+1, m-i+1}$, where $i,j=\set{1,\dots,m}$.
}

\headerbox{Adjacency matrix}{name=adj mat, column=0,row=2,span=1,height=0.4, below=def}{
\begin{definition}[Adjacency]
Two combinations $J$ and $K$ with the same cardinality are said to be adjacent if they differ by one element.
That is,
$$|K \cap J| = |K|-1=|J|-1.$$
\end{definition}

Let $A_m^n$ be the adjacency matrix for $m$-combinations taken over $n$ elements.
The following lemma gives a recurrence relation to construct $A_m^n$.

\begin{lemma} \label{lem: recurrence}
For $n > m > 1$
\begin{align*}
A_m^n = & \begin{bmatrix} A^{n-1}_m & \tilde{A}_m^n \\ (\tilde{A}_m^n)^\top & A_{m-1}^{n-1} \end{bmatrix}, \\
\tilde{A}_m^n = & \begin{bmatrix} \tilde{A}_m^{n-1} & 0_{\binom{n-2}{m-2} \times \binom{n-2}{m}} \\ I_{\binom{n-2}{m-1}} & \tilde{A}_{m-1}^{n-1} \end{bmatrix},
\end{align*}
with starting conditions
\begin{align*}
A_1^n = & \begin{bmatrix} 0 & 1 & \dots & 1 \\ 1 & 0 & \ddots & \vdots \\ \vdots & \ddots & \ddots & 1 \\ 1 & \dots & 1 & 0 \end{bmatrix}, &
%\tilde{A}_1^n = & \begin{bmatrix} 1 \\ \vdots \\ 1 \end{bmatrix}, &
\tilde{A}_2^4 = & \begin{bmatrix} 1 & 1 & 0 \\ 1 & 0 & 1 \\ 0 & 1 & 1 \end{bmatrix} .
\end{align*}
\end{lemma}
}

\headerbox{Proof}{name=proof, column=1, row=2,span=2, height=0.4, below=rank}{
\begin{multicols}{2}

\begin{center}
\begin{tabular}{r | c c}
	& $J_i$ & $J_j$ \\ \hline
	$K_i$ & $A^{n-1}_m$ & $\tilde{A}^n_m$ \\
	$K_j$ & $(\tilde{A}^n_m)^\tau$ & $A^{n-1}_{m-1}$
\end{tabular}
\end{center}

The combinations $J_i$ and $K_i$ are $m$-combinations over $n-1$ elements, so the upper left block is $A_m^{n-1}$:
\begin{equation*}
\left ( A_m^{n-1} \right )_{N(J_i)+1,N(K_i)+1} = \left ( A_m^n \right )_{N(J_i)+1,N(K_i)+1}.
\end{equation*}

Both $J_j$ and $K_j$ have the element $n-1$, so
\begin{align*}
N(J_j) = & \binom{n-1}{m} + N(J_j \setminus \set{n-1}), \\
N(K_j) = & \binom{n-1}{m} + N(K_j \setminus \set{n-1}).
\end{align*}
This makes the bottom right block $A_{m-1}^{n-1}$.

Subdivide the combinations $J_i$ and $K_j$ into sets that do or do not contain $n-1$ or $n-2$.
\begin{center}
\begin{tabular}{r | c c}
	& $n-1 \notin$ & $n-1 \in$ \\ \hline
	$n-2 \notin$ & $\tilde{K}_i$ & $\tilde{J}_i$ \\
	$n-2 \in$ & $\tilde{K}_j$ & $\tilde{J}_j$
\end{tabular}
\end{center}
$\tilde{K}_i$ and $\tilde{J}_j$ differ by at least two elements, so the relevant block is a zero matrix.
$\tilde{J}_i$ and $\tilde{K}_j$ are adjacent if and only if $\tilde{J}_i \setminus \set{n-1} = \tilde{K}_j \setminus \set{n-2}$.
This makes the relevant block an identity matrix.
$\tilde{J}_j$ and $\tilde{K}_j$ are adjacent if and only if $\tilde{J}_j \setminus \set{n-1}$ and $\tilde{K}_j \setminus \set{n-2}$ are adjacent.
The relevant block is then a copy of $\tilde{A}_{m-1}^{n-1}$.

For the final block, consider a new set $J_k = \tilde{J}_i \setminus \set{n-1} \cup \set{n-2}$.
Then $\tilde{J}_i$ is adjacent to $\tilde{K}_i$ if and only if $J_k$ is adjacent to $\tilde{K}_i$.
Thus,
\begin{equation*}
\left ( A_m^n \right )_{N(\tilde{J}_i)+1,N(\tilde{K}_i)+1} = \left ( A_m^{n-1} \right )_{N(J_k)+1,N(\tilde{K}_i)+1}.
\end{equation*}
The block is then a copy of $\tilde{A}_m^{n-1}$.

Since adjacency is bidirectional, $A_m^n$ is symmetric, and the bottom left block is the transpose of the upper right block.

The following corollary means only half of all $A_m^n$ need to be found.
\begin{corollary} \label{lem: anti-transpose}
\begin{equation*}
(A_m^n)^\tau = A_{n-m}^n.
\end{equation*}
\end{corollary}
\end{multicols}
}

\headerbox{Union of adjacent combinations}{name=add, column=0,row=3,span=2,height=0.25,below=adj mat}{

Define a matrix $B_m^n$ that stores the values of $N(J \cup K)$ for when $J$ and $K$ are adjacent:
\begin{equation*}
\left ( B^n_m \right )_{N(J) + 1, N(K)+1} = \begin{cases} N(J \cup K)+1 & J,K \text{ adjacent} \\ 0 & \text{otherwise} \end{cases}
\end{equation*}

\begin{multicols}{2}

\begin{lemma}
\begin{align*}
B^n_1 = & \begin{bmatrix} 0 & 1 & 2 & \dots & \binom{n}{2}+1 \\
							& 0 & 3 & \dots & \binom{n}{2}+2 \\
							&    & \ddots & \dots & \vdots \\
							&	&	&	0	&	\binom{n}{2}+n \\
							&	&	&		&	0 \end{bmatrix}, &
B^n_{n-1} = & A^n_1.
\end{align*}
\end{lemma}

\begin{lemma}
\begin{align*}
B^n_m = & \begin{bmatrix} B^{n-1}_m & \tilde{B}^n_m \\ ( \tilde{B}^n_m )^\top & B^{n-1}_{m-1} \end{bmatrix} + \binom{n-1}{m+1} \begin{bmatrix} 0 & \tilde{A}^n_m \\ 0 & A^{n-1}_{m-1} \end{bmatrix}, \\
\tilde{B}^n_m = & \begin{bmatrix} \tilde{B}^{n-1}_m & 0 \\ \diag(K) & \tilde{B}^{n-1}_{m-1} \end{bmatrix} - \binom{n-2}{m+1} \begin{bmatrix} \tilde{A}^{n-1}_m & 0 \\ 0 & 0 \end{bmatrix},
\end{align*}
where $\diag(K)$ is the diagonal matrix with entries equal to the row index.
\end{lemma}

\end{multicols}
}
\headerbox{Intersections}{name=sub,column=2,span=1,height=0.25,below=proof}{
\begin{corollary}
Let $J$ and $K$ be two adjacent $m$--combinations, then
$$\left ( B^n_m \right )^\tau_{N(J)+1,N(K)+1} = \binom{n}{m} - N(J \cap K) .$$
\end{corollary}

\begin{align*}
\left ( B^n_m \right)^\tau_{N(J)+1,N(K)+1} = & \left ( B^n_m \right)_{\binom{n}{m} - N(K), \binom{n}{m} - N(J)} \\
	= & \left ( B^n_m \right)_{N(K^\complement)+1, N(J^\complement) + 1} \\
	= & N(K^\complement \cup J^\complement)+1 \\
	= & N( (J \cap K)^\complement )+1 \\
	= & \binom{n}{m} - N(J \cap K).
\end{align*}
}

\nocite{lehmer1964machine, mccaffrey2004generating, donnot2020unranking, macaulay1927some}

\headerbox{References}{name=ref,column=0,span=3,below=add}{
\begin{multicols}{2}
\bibliographystyle{siam}
\renewcommand{\section}[2]{\vskip 0.05em}
\bibliography{REF_Combinadics_v1_20230601}
\end{multicols}
}
 
\end{poster}

\end{document}